\documentclass{article}
\usepackage[utf8]{inputenc}
\usepackage{listings}
\usepackage{color}
\usepackage{hyperref}
\usepackage{graphicx}
\usepackage{float}

\definecolor{codegreen}{rgb}{0,0.6,0}
\definecolor{codegray}{rgb}{0.5,0.5,0.5}
\definecolor{codepurple}{rgb}{0.58,0,0.82}
\definecolor{backcolour}{rgb}{0.95,0.95,0.92}

\lstdefinestyle{mystyle}{
    backgroundcolor=\color{backcolour},   
    commentstyle=\color{codegreen},
    keywordstyle=\color{magenta},
    numberstyle=\tiny\color{codegray},
    stringstyle=\color{codepurple},
    basicstyle=\ttfamily\footnotesize,
    breakatwhitespace=false,         
    breaklines=true,                 
    captionpos=b,                    
    keepspaces=true,                 
    numbers=left,                    
    numbersep=5pt,                  
    showspaces=false,                
    showstringspaces=false,
    showtabs=false,                  
    tabsize=2
}

\lstset{style=mystyle}

\title{LinkedIn Data Analysis: IIT Graduates Career Paths}
\author{Data Analytics Team}
\date{\today}

\begin{document}

\maketitle

\section{Introduction}
This document outlines our comprehensive approach to analyzing IIT graduates' career paths using LinkedIn data. The analysis includes data collection methodology, insights generation, and implementation approach.

\section{Data Collection Implementation}
Below is the Python implementation for collecting LinkedIn data:

\begin{lstlisting}[language=Python]
import os
from linkedin_api import Linkedin
import pandas as pd
from time import sleep
from datetime import datetime
import logging

class LinkedInDataCollector:
    def __init__(self, client_id, client_secret):
        self.logger = self._setup_logging()
        self.api = None
        self.client_id = client_id
        self.client_secret = client_secret
        
    def _setup_logging(self):
        logging.basicConfig(
            level=logging.INFO,
            format='%(asctime)s - %(levelname)s - %(message)s',
            handlers=[
                logging.FileHandler('linkedin_scraper.log'),
                logging.StreamHandler()
            ]
        )
        return logging.getLogger(__name__)

    def authenticate(self):
        """Authenticate with LinkedIn using OAuth2"""
        try:
            self.api = Linkedin(self.client_id, self.client_secret)
            self.logger.info("Successfully authenticated with LinkedIn API")
        except Exception as e:
            self.logger.error(f"Authentication failed: {str(e)}")
            raise

    def search_profiles(self, keywords, limit=100):
        """Search for profiles matching specific keywords"""
        profiles = []
        try:
            search_results = self.api.search_people(
                keywords=keywords,
                limit=limit
            )
            
            for profile in search_results:
                profile_data = {
                    'profile_id': profile.get('public_id'),
                    'name': profile.get('full_name'),
                    'headline': profile.get('headline'),
                    'company': profile.get('current_company'),
                    'industry': profile.get('industry'),
                    'location': profile.get('location'),
                    'education': self._extract_education(profile)
                }
                
                if self._is_iit_graduate(profile_data['education']):
                    profiles.append(profile_data)
                
                # Respect rate limits
                sleep(1)
                
        except Exception as e:
            self.logger.error(f"Error during profile search: {str(e)}")
            
        return profiles

    def _is_iit_graduate(self, education):
        """Check if the profile is from an IIT"""
        if not education:
            return False
            
        iit_keywords = ['indian institute of technology', 'iit']
        return any(any(keyword in str(edu).lower() for keyword in iit_keywords) 
                  for edu in education)

    def _extract_education(self, profile):
        """Extract education details from profile"""
        education = []
        if 'education' in profile:
            for edu in profile['education']:
                education.append({
                    'school': edu.get('school_name'),
                    'degree': edu.get('degree_name'),
                    'field': edu.get('field_of_study'),
                    'year': edu.get('end_date', {}).get('year')
                })
        return education

    def save_to_csv(self, profiles, filename=None):
        """Save collected data to CSV"""
        if not filename:
            filename = f'linkedin_data_{datetime.now().strftime("%Y%m%d_%H%M%S")}.csv'
            
        try:
            df = pd.DataFrame(profiles)
            df.to_csv(filename, index=False)
            self.logger.info(f"Data successfully saved to {filename}")
        except Exception as e:
            self.logger.error(f"Error saving data to CSV: {str(e)}")
def main():
    # Replace with LinkedIn API credentials
    CLIENT_ID = "client_id"
    CLIENT_SECRET = "client_secret"
    
    collector = LinkedInDataCollector(CLIENT_ID, CLIENT_SECRET)
    
    try:
        # Authenticate
        collector.authenticate()
        
        # Search for IIT graduates
        profiles = collector.search_profiles(
            keywords="IIT graduate",
            limit=100
        )
        
        # Save results
        collector.save_to_csv(profiles)
        
    except Exception as e:
        print(f"An error occurred: {str(e)}")

if __name__ == "__main__":
    main()
\end{lstlisting}

\section{Data Analysis Methodology(Bonus)}
The collected data will be analyzed to generate the following insights:

\subsection{Career Trajectory Analysis}
\begin{itemize}
    \item Track progression patterns between job titles
    \item Identify typical duration at each career level
    \item Analyze paths leading to leadership positions
    \item Map industry transition patterns
\end{itemize}

\subsection{Industry Patterns}
\begin{itemize}
    \item Determine dominant sectors for IIT graduates
    \item Correlate specializations with industry choices
    \item Identify emerging industry trends
    \item Calculate industry retention rates
\end{itemize}

\subsection{Company Analysis}
\begin{itemize}
    \item Compare startup vs. established company preferences
    \item Analyze company size transitions
    \item Identify companies with best retention rates
    \item Map career growth patterns by company type
\end{itemize}

\section{Implementation Approach}
Our implementation follows these key principles:

\subsection{Architecture}
\begin{itemize}
    \item Object-oriented design for maintainability
    \item Comprehensive error handling and logging
    \item Rate limiting for API compliance
    \item Scalable data processing
\end{itemize}

\subsection{Key Features}
\begin{itemize}
    \item Secure OAuth2 authentication
    \item Efficient profile searching
    \item Structured data extraction
    \item Education verification
    \item CSV export functionality
\end{itemize}

\subsection{Safety \& Compliance}
\begin{itemize}
    \item API-based access
    \item Built-in rate limiting
    \item Error logging system
    \item Terms of service compliance
    \item Secure credential management
\end{itemize}

\subsection{Data Quality}
\begin{itemize}
    \item Structured collection methods
    \item Missing data handling
    \item Education validation
    \item Consistent formatting
\end{itemize}

\end{document}